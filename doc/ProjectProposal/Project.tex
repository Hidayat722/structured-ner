\documentclass[a4paper,10pt]{article}
%\documentclass[a4paper,10pt]{scrartcl}

\usepackage[utf8]{inputenc}

\title{Project Proposal Machine Learning}
\author{}
\date{}

\usepackage{hyperref}
\usepackage{xcolor}
\definecolor{dark-red}{rgb}{0.4,0.15,0.15}
\definecolor{dark-blue}{rgb}{0.15,0.15,0.4}
\definecolor{medium-blue}{rgb}{0,0,0.5}
\hypersetup{
    colorlinks, linkcolor={dark-red},
    citecolor={dark-blue}, urlcolor={medium-blue}
}



\begin{document}
\maketitle

% 
% 
% > 1) Title project 
% > 2) Student member names with student numbers 
% > 3) Application, explain how you get data or simulator, how many examples, how 
% > many inputs, outputs, whether it is an RL, classification, or regression problem. 
% > 4) Methods. Explain the methods that you will compare. Identify the learning 
% > algorithm and the optimization method. 


\section*{Project Proposal}

\paragraph{Project title:}
Structured Prediction for Named Entity Recognition

\paragraph{Student member names \& number:\\\\}

\begin{tabular}{l l}
Joachim Daiber   & 2397331 \\
Carmen Klaussner & 2401541\\
\end{tabular}



\paragraph{Application:}

Named Entity Recognition is a classification problem in which the goal is to correctly predict the named entity types for
the tokens in a text.


We will obtain our training data from the CoNLL-2003 dataset \cite{TjongKimSang:2003:ICS:1119176.1119195}, which contains
data in English and German. Additionally, we may also use the CoNLL-2002 data for Dutch and Spanish.  Table~\ref{table:input} shows the size of the input data and Table~\ref{table:output} shows the types and counts of Named Entity annotations on the input data for German and English.



\begin{table}[h!]

\centering
 \begin{tabular}{l|l|l|l}
 \textbf{English Data}& Articles & Sentences & Tokens \\ \hline
 Training set &   946       &   14,987        &   203,621     \\
 Development set& 216        &  3,466         &    51,362     \\
 Test set       & 231        &   3,684        &     46,435    \\
  
 \end{tabular}

\vspace*{0.2cm}

 \begin{tabular}{l|l|l|l}
 \textbf{German Data} & Articles & Sentences & Tokens \\ \hline
 Training set &     553     &  12,705         &  206,931      \\ 
 Development set&    201     &  3,068         &  51,444       \\
 Test set       &    155     &   3,160        &   51,943      \\
  
 \end{tabular}
\caption{Sizes of the training, development and test sets.}
\label{table:input}

\end{table}


\begin{table}[h!]

\centering
 \begin{tabular}{l|l|l|l|l}
 \textbf{English Data} & LOC      & MISC     & ORG &PER       \\ \hline
 Training set &   7140       &  3438  & 6321     & 6600    \\
 Development set&  1837       &  922         &  1341    &  1842    \\
 Test set       &  1668       &   702        &   1661    &  1617   \\

 \end{tabular}

\vspace*{0.2cm}

\centering
 \begin{tabular}{l|l|l|l|l}
 \textbf{German Data} & LOC      & MISC     & ORG &PER       \\        \hline
 Training set &   4363       &   2288        & 2427     & 2773    \\
 Development set&  1181       &   1010        &  1241    &  1401    \\
 Test set       &   1035      &    670       &  773     &  1195   \\
  
 \end{tabular}
\caption{Named Entity annotations in the training, development and test sets.}
\label{table:output}

\end{table}




\paragraph{Methods:}
We believe that Structured Prediction provides a flexible and efficient model for Named Entity Recognition (see \cite{strlearn}).
The learning algorithm is the Structured Perceptron with Averaging \cite{Collins:2002:DTM:1118693.1118694}.



% > 6) Chosen programming language 
% > 7) Planning. Describe the different phases and the planning deadlines (until November).
% 
\paragraph{Setup of Experiments:}
We will compare multiple sets of features on the given training and test set, 
while trying to find a good balance between complexity and performance.


\paragraph{Chosen programming language:}
Python with the NumPy package


\section*{Planning}

\begin{table}[h!]
\begin{tabular}{l l}
17-23 Sep. & Data Preparation and Literature Review \\
24-30 Sep. & Implementation of Learning Algorithm\\
1-7 Oct.   & Implementation of Decoding\\
8-14 Oct.  & Improvement of Features\\
15-21 Oct. & Evaluation \\
22-28 Oct. & Paper \\
29-4 Nov.  & Paper\\
\end{tabular}
\end{table}




{\small
\bibliographystyle{plain}
\bibliography{ML.bib}
}


\end{document}
