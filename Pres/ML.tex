\documentclass[10pt]{beamer}
\mode<presentation>
% \mode<handout>
{
  %\usetheme{CambridgeUS}
  \setbeamercovered{transparent}

  \definecolor{rugcolor}{rgb}{0,0.3,0.6}

  \beamertemplatenavigationsymbolsempty
  \setbeamercolor{item}{fg=rugcolor!60!black}
  \setbeamercolor{title}{rugcolor!60!black}
  \setbeamercolor{frametitle}{rugcolor!60!black}

%HEADER WITH HIGHLIGHTED SECTION NAMES (optional)
  \useheadtemplate{%
    \vbox{%
        %\vskip1.2pt
        %\pgfuseimage{logo}
    %\vskip1.2pt
    \tinycolouredline{rugcolor}{\color{white}{\bf
\insertsectionnavigationhorizontal{\paperwidth}{}{\hskip0pt plus1filll
% \pgfuseimage{logo}
}}}
\hskip0pt plus1filll
\tinycolouredline{rugcolor}{\color{white}{\bf
\insertsubsectionnavigationhorizontal{\paperwidth}{}{\hskip0pt plus1filll
% \pgfuseimage{logo}
}}}
\flushright
\vskip1.0pt
 \pgfuseimage{logo}
}
}

% %FOOTER WITH AUTHOR NAME(S), PAPER TITLE (ABBREVIATED IF SPECIFIED BY \title),
% % AND PAGE COUNTER (optional)
%   \usefoottemplate{%
%     \vbox{%
%     \tinycolouredline{rugcolor}%
%         {\color{white}{{\insertshortauthor}\hfill%
%            \insertshorttitle\hfill%
%            \textsc{\insertframenumber/\inserttotalframenumber}
%           }}
%         }
% 
%       }
}

\pgfdeclareimage[height=0.5cm]{logo}{logo2.jpg}

\usepackage[english]{babel}
% or whatever

\usepackage[latin1]{inputenc}
% or whatever

%\usepackage{times}
%\usepackage[T1]{fontenc}
% Or whatever. Note that the encoding and the font should match. If T1
% does not look nice, try deleting the line with the fontenc.


\title[Machine Learning] % (optional, use only with long paper titles)
{
Structured Prediction for Named Entity Recognition}

% \subtitle
% {Week 1} % (optional)

\author % (optional, use only with lots of authors)
{Joachim Daiber \& Carmen Klaussner}
% - Use the \inst{?} command only if the authors have different
%   affiliation.

 \institute[University of Groningen] % (optional, but mostly needed)
 {
  Information Science\\
  University of Groningen
  }
% - Use the \inst command only if there are several affiliations.
% - Keep it simple, no one is interested in your street address.

\date[Short Occasion] % (optional)
{24th. October, 2012}

%\subject{Talks}
% This is only inserted into the PDF information catalog. Can be left
% out. 



\begin{document}

\begin{frame}
  \titlepage
\end{frame}

%%%%%%%%%%%%%%%%%%%%%%%%%%%%%%%%%%%%%%%%%%%%%%%%%%%%%%


%\section{}
\begin{frame}
\frametitle{Motivation}

% Named Entity a lot of practical uses/applications

\end{frame}



%%%%%%%%%%%%%%%%%%%%%%%%%%%%%%%%%%%%%%%%%%%%%%%%%%%%




\begin{frame}
  \frametitle{Outline}
  \tableofcontents
  % You might wish to add the option [pausesections]
\end{frame}




%%%%%%%%%%%%%%%%%%%%%%%%%%%%%%%%%%%%%%%%%%%%%%%%%

\section{Introduction}
\begin{frame}
\frametitle{Named Entity Recognition (NER)}

The task of \textbf{Named Entity Recognition} divides into 
two subtasks: \\
\begin{itemize}
 \item Named Entity Detection 
 \item Named Entity Classification
\end{itemize}
\end{frame}


%%%%%%%%%%%%%%%%%%%%%%%%%%%%%%%%%%%%%%%%%%%%%%%%%%%%%


%\section{}
\begin{frame}
\frametitle{Entity Classes in NER}

\begin{itemize}
 \item Enamex types
\begin{itemize}
 \item Person Names: \texttt{John Bateman}
 \item Organisations: \texttt{Lavazza}
 \item Locations: \texttt{France, Bristol}
\end{itemize}

 \item Miscellaneous (CoNLL)
 \begin{itemize}
  \item  proper names outside the classic \emph{enamex}
  \item the type \emph{product}
 \end{itemize}

 \item timex (Date \& Time Expressions)
 
\item numex Monetary Values \& Percent
\end{itemize}	

$\Rightarrow$ only specific entities; \emph{in June}/ \emph{the prof} (undefined year/person) % deictic reference to sth. in previous text 
 \end{frame}

 
%%%%%%%%%%%%%%%%%%%%%%%%%%%%%%%%%%%%%%%%%%%%%%%%%%%%%%%%


%\section{}
\begin{frame}
\frametitle{Approaches to NER}

\begin{enumerate}
 \item linguistic grammar-based techniques
 \item statistical models
\end{enumerate}

1. hand-crafted rules may obtain a high precision, but at cost of low recall
and extensive work by computational linguists\\

2. Statistical NER systems require large amount of manually annotated training data

$\Rightarrow$ supervised methods most prominent


\end{frame}


%%%%%%%%%%%%%%%%%%%%%%%%%%%%%%%%%%%%%%%%%%%%%%%%%%%%%%%%%					
%\section{}
\begin{frame}
\frametitle{Issues for NER}


\begin{itemize}
 \item Ambiguity


\begin{itemize}
 \item \textbf{Polysemy:} \texttt{Location} vs. \texttt{Person} \\
 
 \emph{Paris (France) - Paris (Hilton)}\\
 
 \item \textbf{Metonymy:}  (part-whole):
 
 ``\textbf{Paris} has decided to introduce an increase in tax...'' \\
 
 $\Rightarrow$ (the \textbf{government} not the \textbf{city})  
\end{itemize}

 \item mainly domain-specific systems - not readily portable to different domain/genre % drop in performance for every system (some 20% to 40% of precision and recall).
 
\end{itemize}

\end{frame}


%%%%%%%%%%%%%%%%%%%%%%%%%%%%%%%%%%%%%%%%%%%%%%%%%%%%%%%%%		

\section{Our Structured Perceptron}
\begin{frame}
\frametitle{Training and Test Data}

\begin{itemize}
 \item CoNLL 2002/2003 
 \item Languages: English, German, Spanish \& Dutch
\end{itemize}
\end{frame}


%%%%%%%%%%%%%%%%%%%%%%%%%%%%%%%%%%%%%%%%%%%%%%%%%%%%%%%%%		


%\section{Our Structured Perceptron}
\begin{frame}
\frametitle{}
\end{frame}

%%%%%%%%%%%%%%%%%%%%%%%%%%%%%%%%%%%%%%%%%%%%%%%%%%%%%%%%%%%


%\section{}
\begin{frame}
\frametitle{}
\end{frame}


%%%%%%%%%%%%%%%%%%%%%%%%%%%%%%%%%%%%%%%%%%%%%%%%%%%%%%%%%












%%%%%%%%%%%%%%%%%%%%%%%%%%%%%%%%%%%%%%%%%%%%%%%%%%%%%%%%%		

%% Template

% \section{}
% \begin{frame}
% \frametitle{}
% \end{frame}


%%%%%%%%%%%%%%%%%%%%%%%%%%%%%%%%%%%%%%%%%%%%%%%%%%%%%%%%


\begin{frame}[allowframebreaks]
        \frametitle{References}
        \nocite{*}
        \bibliographystyle{plain}
        \bibliography{ML}
\end{frame}



\end{document}
